\chapter[Análise de Sentimentos]{Análise de Sentimentos} \label{cap:cap1}
Informações textuais - ou informações que podem ser representadas de forma textual - podem ser divididas em duas categorias principais: fatos e opiniões. Fatos são expressões objetivas sobre entidades, eventos e suas propriedades, enquanto opiniões são usualmente expressões subjetivas que descrevem o sentimento em relação a entidades, eventos e suas propriedades.\cite{LiuSentAnaSubject} \par

Em quase todos cenários relacionados com tomada de decisões e percepção da realidade, as experiências e opiniões de outras fontes condicionam nosso meio de ação. Dessa forma, tanto pessoas como organizações buscam por opiniões ou críticas  para ponderar suas possibilidades. Nesta conjectura, conceitos relacionados como sentimentos, avaliações, atitudes e emoções são os principais assuntos do estudo da análise de sentimentos, campo que estuda como as informações são organizadas e representam a opinião - ou sentimento - do escritor. \cite{LiuBing} \par

Estudos no campo da psicologia social afirmam que  pessoas e organizações usualmente se apoiam em decisões de terceiros para tomarem suas próprias. As pessoas podem se conformar com opiniões de outros por seu desejo de fazer escolhas corretas sem depender da incerteza ou devido ao desejo de serem aceitas num meio convencional e padronizado. Outra possibilidade sugere que as pessoas simplesmente imitam o comportamento de seus semelhantes, uma vez que a imitação pode aumentar a eficiência na tomada de decisões, permitindo que ações que obtiveram sucesso sejam repetidas. De fato, estudos comprovam que as organizações tendem a incorporar decisões que constituem um padrão ou tendência. O filosofo estadunidense Daniel Dennett inferiu em 1995 que a sociedade forma sua cultura ao adotar as inovações de outras sociedades.\cite{PeopleLikeThis} \par

Considerando a importância do papel dos sentimentos e opiniões como fatores de construção social e econômica, estes recursos de expressão compõem o objeto central de estudo da análise de sentimentos: um campo amplamente multidisciplinar e abrangente que evolui cada vez mais. \par
% 1a seção do capítulo 1
\section{Contextualização}
\label{sec:sec1}
O campo da análise de sentimentos tem crescido cada vez mais desde os anos 2000 devido principalmente a um aumento na capacidade de armazenamento, processamento e até mesmo geração de dados aumento que se deu devido a evolução das redes sociais e outras formas de compartilhamento de opiniões como fóruns virtuais, blogs, micro-blogs e comentários \cite{LiuBing, LiuSentAnaSubject}.
\par
Este evento acontece por que a internet mudou drasticamente a forma como as pessoas e organizações tomam decisões sobre algo: antes da evolução dos meios de comunicação virtuais, era necessário buscar opinião de um conhecido local para entender determinado produto, enquanto uma empresa precisava conduzir pesquisas de foco e campanhas de conhecimento para saber qual a opinião pública sobre a mesma.Com a tecnologia disponível hoje, grande parte desta informação está disponível na forma de dados não processados nas plataformas previamente citadas. \cite{LiuSentAnaSubject} \par 

No entanto, é importante notar que o enorme volume de dados gerados por essas aplicações é usualmente armazenado no formato bruto de texto e, portanto, não demonstra informações a respeito do conteúdo semântico da mensagem a não ser que sejam interpretados por um humano. Dessa forma, o campo da análise de sentimentos busca estudar tais informações e gerar estatísticas a respeito do valor semântico das mesmas.\cite{LiuBing} \par

Neste sentido, os estudos no campo de análise de sentimentos visam extrair informações sobre as opiniões expressas nos textos não processados tarefa que seria difícil de ser executada por um humano que teria a necessidade de ler todas publicações, estudar sua composição e modelo e por fim inferir o significado de cada uma delas individualmente. Por isso, sistemas de descoberta e análise de opiniões são necessários. A análise de sentimentos cresce dessa necessidade, sendo um desafio de mineração de texto e processamento de linguagem natural que obtém cada vez mais notoriedade devido ao seu alto valor de aplicações práticas como pesquisas de tendências, eleitorais e de opinião pública. \cite{LiuSentAnaSubject} 

\par
